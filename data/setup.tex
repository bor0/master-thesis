% Contents of listings-setup.tex
\usepackage{xcolor}

% Proper line breaks for code with too many chars on a single line (for backticks)
\lstset{
    basicstyle=\ttfamily,
    numbers=left,
    keywordstyle=\color[rgb]{0.13,0.29,0.53}\bfseries,
    stringstyle=\color[rgb]{0.31,0.60,0.02},
    commentstyle=\color[rgb]{0.56,0.35,0.01}\itshape,
    numberstyle=\footnotesize,
    stepnumber=1,
    numbersep=5pt,
    backgroundcolor=\color[RGB]{248,248,248},
    showspaces=false,
    showstringspaces=false,
    showtabs=false,
    tabsize=2,
    captionpos=b,
    breaklines=true,
    breakatwhitespace=true,
    breakautoindent=true,
    escapeinside={\%*}{*)},
    linewidth=\textwidth,
    basewidth=0.5em,
}

\usepackage{data/lstlangdafny}
\usepackage{data/lstlangracket}

\usepackage{graphicx}

% Proper line breaks for code with too many chars on a single line (for verbatim)
\usepackage{fvextra}
\DefineVerbatimEnvironment{Highlighting}{Verbatim}{breaklines,commandchars=\\\{\}}

% Rename captions for figures and tables
\usepackage{caption}
\captionsetup[figure]{name=Слика}
\captionsetup[table]{name=Табела}

% Rename list of figures and tables
\renewcommand{\listfigurename}{Листа на слики}
\renewcommand{\listtablename}{Листа на табели}

% Rename Contents
\renewcommand{\contentsname}{Содржина}

% Some symbols (delta, setminus) are undefined in Times new Roman
\setmathfont{XITSMath-Regular.otf}

% New page after every section
\usepackage{titlesec}
\newcommand{\sectionbreak}{\clearpage}
