\pagenumbering{gobble} % disable page numbers here
\setlength{\parskip}{0.14in}

\begin{center}

\textbf{УНИВЕРЗИТЕТ ЗА ТУРИЗАМ И МЕНАЏМЕНТ ВО СКОПЈЕ}

\textbf{ФАКУЛТЕТ ЗА ИНФОРМАТИКА}

\textbf{Втор циклус постдипломски студии}

\bigbreak

\includegraphics[width=2.01042in,height=2.38542in]{images/utms.jpg}

\vspace{5em}

\textbf{Боро Ситниковски}

Магистерски труд

\textbf{Формална верификација на множество на инструкции во виртуелни машини}

\vspace*{\fill}

\textbf{Скопје, 2020 година}

\newpage{}

\textbf{УНИВЕРЗИТЕТ ЗА ТУРИЗАМ И МЕНАЏМЕНТ ВО СКОПЈЕ}

\textbf{ФАКУЛТЕТ ЗА ИНФОРМАТИКА}

\textbf{Втор циклус постдипломски студии}

\bigbreak

\includegraphics[width=1.11458in,height=1.32292in]{images/utms.jpg}

\end{center}

\vspace{5em}

Студент: \textbf{Боро Ситниковски}

Бр. на индекс: 800003

\bigbreak

Ментор: \textbf{Проф. д-р Билјана Стојчевска}

\begin{center}

\vspace*{\fill}

\textbf{Скопје, 2020 година}

\newpage{}

\textbf{ФАКУЛТЕТ ЗА ИНФОРМАТИКА}

\end{center}

\vspace{5em}

Магистерски труд

\textbf{Формална верификација на множество на инструкции во виртуелни машини}

\vspace{10em}

Автор: \\ \indent \textbf{Боро Ситниковски}

Ментор: \\ \indent \textbf{Проф. д-р Билјана Стојчевска}

\vspace*{\fill}

Комисија за оценка на магистерскиот труд:

\begin{enumerate}
\def\labelenumi{\arabic{enumi}.}
\item
  Проф. д-р Лидија Горачинова-Илиева
\item
  Проф. д-р Симе Арсеновски
\item
  Проф. д-р Славчо Чунгурски
\end{enumerate}

\newpage{}

\begin{quote}
\textbf{Формална верификација на множество на инструкции во виртуелни машини}

\vspace{0.14in}

\textbf{Апстракт}

\vspace{0.14in}

\hspace*{10mm} Трудот претставува истражување на формална верификација на множество на инструкции во виртуелни машини, со помош на автоматскиот докажувач на теореми Dafny и со математички докази на коректност. Ќе биде покажано дека со формална верификација на компјутерски системи (претставени преку виртуелни машини), се олеснува анализата на истите а со тоа се олеснува и можноста да се подобрат истите.

\hspace*{10mm} Ќе бидат претставени формални системи, математичка логика, теорија на множества, како и Геделовите теореми на нецелосност.

\hspace*{10mm} Потоа ќе бидат претставени Тјурингови машини, како и нивните ограничувања во однос на Геделовите теореми за нецелосност. Исто така ќе се претстави Хоаровата логика која е теорија што дозволува да се изразат математички тврдења (докази) за развиените програми. Преку примери ќе биде покажано како овие системи можат да се користат за докажување на одредени својства за програми.

\hspace*{10mm} Конечно, ќе бидат разгледани виртуелни машини, а потоа ќе биде претставена пример имплементација за виртуелна машина и ќе се покаже како може да се верификува нејзиното множество на инструкции во Dafny.

\hspace*{10mm} Во прилог ќе се покаже поврзаноста помеѓу обработка на симболи и самореференција (Геделови теореми), а потоа ќе се претстави минимален систем за докази како и едноставен пример за автоматски докажувач на теореми и програмирање со ограничувања, за да се долови пример за поедноставена имплементација на програмски јазик налик Dafny.

\hspace*{10mm} Резултатите од истражувањето се и тоа како корисни во денешното дигитално време, каде компјутерските системи се наоѓаат во секој дел од секојдневниот живот. Формални докази за точност на овие софтверски и хардверски системи се круцијални за тие да функционираат како што се очекува.

\vspace{0.14in}

\hspace*{10mm} \textbf{Клучни зборови:} Математички докази, Dafny, виртуелни машини, множество на инструкции, формална верификација

\newpage{}

\textbf{Formal verification of Instruction Sets in Virtual Machines}

\vspace{0.14in}

\textbf{Abstract}

\vspace{0.14in}

\hspace*{10mm} This thesis represents the research of formal verification of instruction sets in virtual machines, with the help of the automated theorem prover Dafny and with mathematical proofs of correctness. It will be shown that with formal verification of computer systems (represented by virtual machines), analysis and the possibility to improve these systems is made easier.

\hspace*{10mm} Presented will be formal systems, mathematical logic, and set theory, as well as Gödel's incompleteness theorems.

\hspace*{10mm} Further, Turing Machines will be introduced, as well as their limitations with respect to Gödel's incompleteness theorems. Hoare logic, which is the theory that allows for expressing mathematical claims (proofs) about programs, will be represented as well. With examples, it will be shown how these systems can be used to prove certain properties about programs.

\hspace*{10mm} Finally, virtual machines will be introduced, together with an example implementation. It will be shown how its instruction set can be verified with Dafny.

\hspace*{10mm} In addition, the connection between symbolic manipulation and self-reference (Gödel's theorems) will be shown, and then a minimal proof system will be presented, as well as a simple example of an automated theorem prover and constraint programming, to capture a simple implementation similar to Dafny.

\hspace*{10mm} The research results are useful in today's digital age, where computer systems are found in every part of daily life. Formal proofs of the correctness of these software and hardware systems are crucial for them to function as expected.

\vspace{0.14in}

\hspace*{10mm} \textbf{Keywords:} Mathematical proofs, Dafny, virtual machines, instruction set, formal verification
\end{quote}

\newpage{}

\setlength{\parskip}{0pt}
